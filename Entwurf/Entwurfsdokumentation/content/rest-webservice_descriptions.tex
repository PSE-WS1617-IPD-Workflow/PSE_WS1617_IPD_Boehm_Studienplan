\section{REST-Webservice Spezifikation}
\subsection{Konvention}
Atomare Werte werden mittels \jsonatom{name} angegeben, wobei name ein eindeutiger Bezeichner für den Wert ist.\\
Zusammengesetzte JSON-Datenklassen werden mittels \jsonobj{name} angegebenm wobei name ein eindeutiger Bezeichner für das JSON-Datenklassen ist.\\
Teile einer zusammengesetzten JSON-Datenklasse werden mittels \jsonpart{name}{prop1, prop2, prop3} angegeben, wobei prop1, prop2, prop3 die einzigen Werte sind, welche angezeigt werden.
% restusecase environment
% #1: Title of the use case, like "A{n}: Title"
\newenvironment{restusecase}[1]{
	\begin{longtable}{|p{.43\linewidth}|p{.55\linewidth}|}
		\multicolumn{2}{@{}l}{\textsf{\textbf{#1}}} \vspace{2pt} \\
		\hhline{|=|=|}
		\textbf{Aufruf} & \textbf{Antwort} \\
		\hline 
		\endfirsthead
		
		\hline
		\textbf{Aufruf} & \textbf{Antwort} \\
		\endhead
		
		\hhline{|=|=|}
		\endlastfoot 
}{
	\end{longtable} \vspace{-12pt}
}

\newcommand{\case}[1]{\textit{#1}}

%%%%%%%%%%%%%%%%%%%%%%%%%%%%%%%%%%%%%%%%%%%%%%%%%%%%%%%%%%%%%

\FloatBarrier
\subsection{Umsetzung der Anwendungsfälle mittels REST-Kommunikation}

TODO: Zitiere Pflichtenheft? Oder irgendein allg. Hinweis?

\begin{restusecase}{A10: Erstanmeldung}
	GET /auth/login
	& (Weiterleitung) \\
	\hline
	GET /student
	& (Statuscode für Erstanmeldung) \\
	\hline
	GET /filters
	& [\jsonobj{Filter}, ...] \\
	\hline
	\multicolumn{2}{|c|}{(Bei Bedarf: Modulfilterung (s. A110))} \\
	\hline
	PUT /student \newline \jsonobj{Student}
	& (Bestätigung) \\
\end{restusecase}

\begin{restusecase}{A20: Login}
	GET /auth/login
	& (Weiterleitung) \\
	\hline
	GET /student
	& \jsonobj{Student} \\
	\hline
	GET /plans
	& [\jsonobj{Studienplan}]
\end{restusecase}

\begin{restusecase}{A30: Logout}
	(Session beenden? Wie?) & 
\end{restusecase}

\begin{restusecase}{A40: Profil bearbeiten}
	GET /student
	& \jsonobj{Student} \\
	\hline
	GET /filters
	& [\jsonobj{Filter}, ...] \\
	\hline
	\multicolumn{2}{|c|}{(Bei Bedarf: Modulfilterung (s. A110))} \\
	\hline
	PUT /student \newline \jsonobj{Student}
	& (Bestätigung) \\
\end{restusecase}

\begin{restusecase}{A50: Neuen Studienplan anlegen}
	POST /plans \newline \jsonatom{Studienplan-Name}
	& (Bestätigung/Fehler) \\
	\hline
	GET /filters
	& [\jsonobj{Filter}, ...] \\
\end{restusecase}

\begin{restusecase}{A55: Studienplan anzeigen}
	GET /plans/\jsonatom{Plan-ID}
	& \jsonobj{Studienplan} TODO: Sind da schon Modulinfos wie Bewertungen mit drin? \\
	\hline
	GET /filters
	& [\jsonobj{Filter}, ...] \\
\end{restusecase}


\begin{restusecase}{A57: Schließen der Studienplan-Ansicht mit Wechsel zur Hauptansicht}
	GET /plans
	& [\jsonpart{Studienplan}{name, id, verified}]
\end{restusecase}

\begin{restusecase}{A60: Studienplan umbenennen}
	PATCH /plans/\jsonatom{Plan-ID} \newline \jsonpart{Studienplan}{name}
	& (Bestätigung/Fehler)
\end{restusecase}

\begin{restusecase}{A70: Studienplan duplizieren}
	POST /plans/\jsonatom{Plan-ID} \newline \jsonpart{Studienplan}{name}
	& (Bestätigung/Fehler)
\end{restusecase}

\begin{restusecase}{A80: Studienplan löschen}
	DELETE /plans/\jsonatom{Plan-ID}
	& (Bestätigung/Fehler)
\end{restusecase}

\begin{restusecase}{A85: Studienplan exportieren}
	GET /plans/\jsonatom{Plan-ID}/pdf
	& (Weiterleitung auf on-the-fly generiertes PDF-Dokument)
\end{restusecase}

\begin{restusecase}{A90: Mehrere Studienpläne duplizieren/löschen/teilen}
	\multicolumn{2}{|c|}{Für alle markierten Studienpläne wird nacheinander aufgerufen:} \\
	\hline
	\case{Duplizieren/Löschen}: s. A70 bzw. A80. \newline
	\case{Teilen}: \newline
	GET /plans/\jsonatom{Plan-ID}/share
	& \case{Duplizieren/Löschen}: (Bestätigung/Fehler) \newline
	\case{Teilen}: [\jsonobj{CopyLink}, ...]
\end{restusecase}

\begin{restusecase}{A100: Vergleichsansicht für Studienpläne}
	\case{Plan 1}: GET /plans/\jsonatom{Plan-ID}
	& \jsonobj{Studienplan} \\
	\hline
	\case{Plan 2}: GET /plans/\jsonatom{Plan-ID}
	& \jsonobj{Studienplan}
\end{restusecase}

\begin{restusecase}{A110: Module in der Suchleiste filtern}
	GET /plan/\jsonatom{Plan-ID}/\+modules\+?\jsonatom{filter}=\jsonatom{value}...
	& [\jsonpart{Modul}{id, name, creditpoints, lecturer, preference}, ...]
\end{restusecase}

\begin{restusecase}{A130: Info-Leiste zu einem Modul anzeigen}
	\multicolumn{2}{|l|}{--- (bereits bei A110 abgerufen).}
\end{restusecase}

\begin{restusecase}{A140: Modul in Studienplan einfügen}
	PUT /plans/\jsonatom{Plan-ID}/\+modules/\+\jsonatom{Modul-ID}  \newline  \jsonpart{Modul}{id, semester}
	& (Bestätigung/Fehler)
\end{restusecase}

\begin{restusecase}{A150: Modul aus Studienplan löschen}
	DELETE /plans/\jsonatom{Plan-ID}/\+modules/\jsonatom{Modul-ID} 
	& (Bestätigung/Fehler)
\end{restusecase}

\begin{restusecase}{A160: Modul innerhalb Studienplan verschieben}
	PATCH /plans/\jsonatom{Plan-ID}/\+modules/\+\jsonatom{Modul-ID}  \newline  \jsonpart{Modul}{semester}
	& (Bestätigung/Fehler)
\end{restusecase}

\begin{restusecase}{A170: Studienplanänderung rückgängig machen}
	\multicolumn{2}{|l|}{(abhängig von letzter Änderung)}
\end{restusecase}

\begin{restusecase}{A180: Modul positiv bewerten / A190: Modul negativ bewerten}
	PUT /plans/\jsonatom{Plan-ID}/\+modules/\+\jsonatom{Modul-ID}/\+preference \newline \jsonpart{Modul}{id, preference}
	& (Bestätigung/Fehler)
\end{restusecase}

\begin{restusecase}{A200: Semester im Studienplan hinzufügen}
	TODO: Semesteranzahl speichern? &
\end{restusecase}

\begin{restusecase}{A210: Semester aus Studienplan löschen}
	TODO: Semesteranzahl speichern? &
\end{restusecase}

\begin{restusecase}{A215: Abgeschlossene Semester im Studienplan ein-/ausblenden}
	\multicolumn{2}{|l|}{---}
\end{restusecase}

\begin{restusecase}{A220: Studienplan auf Korrektheit überprüfen}
	GET /plans/\jsonatom{Plan-ID}/verification 
	& \jsonpart{Studienplan}{id, status, violations}
\end{restusecase}

\begin{restusecase}{A230: Studienplan vervollständigen lassen}
	GET /objective-functions
	& [\jsonobj{Objective}, ...] \\
	\hline
	\multicolumn{2}{|l|}{(Zum Festlegen von Präferenzen siehe A110, A180, A190)} \\
	\hline
	GET /plans/\jsonatom{Plan-ID}/proposal/\+\jsonatom{Zielfunktions-ID}?\jsonatom{prop}=\jsonatom{value}... 
	& \jsonpart{Studienplan}{id, modules} oder Fehlercode. \\
	\hline
	\case{Verwerfen}: --- 
	& --- \\
	\case{Übernehmen}: PUT /plans/\jsonatom{Plan-ID} \newline \jsonobj{Studienplan}
	& (Bestätigung/Fehler) \\
	\case{Unter neuem Namen speichern}: \newline 
	1. POST /plans \newline \jsonatom{Studienplan-Name}
	& (Bestätigung/Fehler) \\
	2. PUT /plans/\jsonatom{Plan-ID} \newline \jsonobj{Studienplan}
	& (Bestätigung/Fehler) \\
\end{restusecase}



% TODO: move into proper paragraph
\begin{lstlisting}[language=json,firstnumber=1]
(*\jsonobj{Student}*) = {
	"studienfach": (*\jsonobj{Studienfach}*),
	"studienbeginn": {
		"turnus": (*\jsonatom{Turnus}*),
		"jahr": (*\jsonatom{Jahr}*)
	},
	"abgeschlossene-module": [(*\jsonobj{Modul}*), ...]
}

(*\jsonobj{Studienfach}*) = {
	"id": (*\jsonatom{Studienfach-ID}*),
	"name": (*\jsonatom{Studienfach-Name}*)
}
\end{lstlisting}

\jsonatom{Turnus} $ \in \{0, 1\}$ mit $0 \equiv \textnormal{SS}$ und $1 \equiv \textnormal{WS}$. \\
\jsonatom{Jahr}: Jahreszahl (Ganzzahl). \\
\jsonatom{Studienfach-ID}: ID, welche ein eindeutiges Studienfach repräsentiert. \\
\jsonatom{Studienfach-Name}: Name des zu \jsonatom{Studienfach-ID} gehörenden Studienfachs. \\
Moduldatentyp
 \begin{lstlisting}[language=json,firstnumber=-3]
 (*\jsonobj{Modul}*) = {
 "menu": {
 "id": (*\jsonatom{Modul-ID}*),
 "semester": (*\jsonatom{Semesternummer}*),
 "name": (*\jsonatom{Modul-Name}*),
 "credit\_points": (*\jsonatom{Modul-Creditpoints}*),
 "preference": (*\jsonatom{Modul-Praeferenz}*),
 "description": (*\jsonatom{description}*),
 "constrains": *\jsonobj{Constrainss}* = {}
 }
 \end{lstlisting}
 Ob ein
\subsection{Authentifizierung}
\label{subsec:api-auth}
\paragraph{Zugrundeliegende Spezifikation} Die Authentifizierung am REST-Webservice erfolgt über eine OAuth 2.0 Schnittstelle nach RFC 6749 (siehe \cite{rfc6749}).
Für das momentane System ist hierbei nur der \textit{Implicit-Grant-Typ} notwendig, jedoch sollte das System mindestens um den \textit{Authorization-Code-Grant-Typ} erweitert werden können.
\paragraph{Registrierung der Klienten}
Wir unterscheiden zwischen zwei Typen von Klienten: vertrauenswürdigen sowie öffentlichen Klienten. Die Definitionen hierzu finden sich auch in RFC 6749. Bei der Webapp unseres Systems handelt es sich auf Grund der Zugänglichkeit des Codes sowie der Daten um einen öffentlichen Klienten.
\subparagraph{Vertrauenswürdiger Klient}
Für einen vertrauenswürdigen Klienten werden folgende Informationen gespeichert:\\
\begin{tabularx}{\textwidth}{@{} | X | X | @{}}
	\hline
	\textbf{Name} & \textbf{Beschreibung}\\ \hline \hline
	api\_key & Eindeutige öffentliche ID des Klienten. Beginnt mit dem Präfix \enquote{key-} \\ \hline
	api\_secret & Nur dem Klienten bekannte Kennung. Beginnt mit dem Präfix \enquote{secret-} \\ \hline
	scope & Berechtigungen des Klienten. Vorerst immer \enquote{student} \\ \hline
	redirect\_url & URL an welche beim \textit{Implicit-Grant} bzw. \textit{Authorization-Code-Grant} weitergeleitet wird. \\ \hline
	origin & Domains, von welchen aus der Klient auf die Ressourcen zugreifen darf. Wird als Regulärer Ausdruck angegeben. \\
	\hline
\end{tabularx}

\subparagraph{Öffentlicher Klient}
Für einen öffentlichen Klienten werden folgende Informationen gespeichert:\\
\begin{tabularx}{\textwidth}{@{} | X | X | @{}}
	\hline
	\textbf{Name} & \textbf{Beschreibung}\\ \hline \hline
	api\_key & Eindeutige öffentliche ID des Klienten. Beginnt mit dem Präfix \enquote{key-} \\ \hline
	api\_secret & NULL (wird nicht benötigt) \\ \hline
	scope & Berechtigungen des Klienten. Vorerst immer \enquote{student} \\ \hline
	redirect\_url & URL an welche beim \textit{Implicit-Grant} bzw. \textit{Authorization-Code-Grant} weitergeleitet wird. \\ \hline
	origin & Domains, von welchen aus der Klient auf die Ressourcen zugreifen darf. Wird als Regulärer Ausdruck angegeben. \\
	\hline
\end{tabularx}
\paragraph{Schnittstellen}
Für die Authentifizierung werden folgende Schnittstellen verwendet:
\begin{table}
	\begin{tabularx}{\textwidth}{@{} | X | X | X | X | @{}}
		\hline
		\textbf{Methode} & \textbf{URL} & \textbf{Beschreibung} & \textbf{Kommunikation} \\ \hline \hline
		GET & /auth/login & Der in RFC 6749, Kapitel 3.1 definierte \textit{Authorization Endpoint} & --- \\ \hline
		POST & /auth/token & Der in RFC 6749, Kapitel 3.2 definierte \textit{Token Endpoint} & --- \\ \hline
	\end{tabularx}
\end{table}
Beim Login wird hierbei der Webbrowser des Benutzers an auf die Seite /auth/login weitergeleitet, auf welcher er sich authentifiziert. Anschließend wird er auf die Hauptseite der Webapp weitergeleitet.
\paragraph{Datenaustausch}
Der Datenaustausch beim Aufruf der Schnittstellen ist wie folgt spezifiziert:\\
\begin{tabularx}{\textwidth}{@{} | X | X | X | @{}}
	\hline
	\textbf{Posten} & \textbf{Implicit-Grant} & \textbf{Authorization-Code-Grant} \\ \hline
	Anfrage-Header &
	\begin{itemize}
		\item[]
		\item[client\_id] \{api\_key\}
	\end{itemize} & \\
\end{tabularx}
\subsection{Atomare Werte}

\subsection{Spezifikation der JSON-Datenklassen}
\begin{lstlisting}[language=json,firstnumber=-3]
(*\jsonobj{Modul}*) = {
	"menu": {
		"id": (*\jsonatom{Modul-ID}*),
		"value": "File",
		"popup": {
			"menuitem": [
				{"value": "New", "onclick": "CreateNewDoc()"},
				{"value": "Open", "onclick": "OpenDoc()"},
				{"value": "Close", "onclick": "blub"}
			]
		}
	},
	"foo" : "bar",
	"bar" : "foo"
}
\end{lstlisting}
\jsonpart{Modul}{menu, foo} wäre die JSON-Datenklasse \jsonobj{Modul} ohne bar.
\subsection{Zugriffsstruktur}

\begin{tabularx}{\textwidth}{@{} | X | X | X | X | @{}}
	\hline
	\textbf{Methode} & \textbf{URL} & \textbf{Beschreibung} & \textbf{\hspace{0pt}Kommunikationsdaten (Note to self: finde schöneres Wort!)} \\ \hline  \hline
	\\ \hline
	& & & \\ \hline
	GET & /student & Lese Informationen über Student & tbd	\\ \hline
	PUT & /student & Ersetze Informationen über Student & tbd
	\\ \hline
	DELETE & /student & Löscht Student & tbd \\ \hline
	& & & \\ \hline
	POST & /plans & Erstellt neuen Studienplan & tbd \\ \hline
	GET & /plans & Lese Planliste & tbd \\ \hline
	GET & /plans/\{plan\_id\} & Lese Plan & tbd \\ \hline
	PUT & /plans/\{plan\_id\} & Ersetze Plan (inkl. Verifikations-Informationen)k & tbd \\ \hline
	PATCH & /plans/\{plan\_id\} & Bearbeite Plan: Semester erstellen, Name ändern, Semester löschen (setzte Verifizierung zurück) & tbd \\ \hline
	DELETE & /plans/\{plan\_id\} & Lösche Plan & tbd \\ \hline
	POST & /plans/\{plan\_id\} & Dupliziere Plan & tbd \\ \hline
\end{tabularx}
\begin{tabularx}{\textwidth}{@{} | X | X | X | X | @{}}
	GET & /plans/\{plan\_id\}/modules & Lese Modul-Liste (gefiltert) & tbd \\ \hline
	GET & /plans/\{plan\_id\}/modules/\{module\_id\} & Lese Modul mit \{module\_id\} & tbd \\ \hline
	PUT & /plans/\{plan\_id\}/modules/\{module\_id\} & Setze Modul in Plan in gegebenes Semester, setzte Verifizierung zurück & tbd \\ \hline
	PATCH & /plans/\{plan\_id\}/modules/\{module\_id\} & Bearbeite Modul in Plan (also setze in anderes Semester), setze Verifizierung zurück & tbd \\ \hline
	DELETE & /plans/\{plan\_id\}/modules/\{module\_id\} & Lösche Modul aus Plan, setzte Verifizierung zurück & tbd \\ \hline
	PUT & /plans/\{plan\_id\}/modules/\{module\_id\}/preference & Setze Bewertung für Modul & tbd \\ \hline
	& & & \\ \hline
	GET & /plans/\{plan\_id\}/verification & Verifiziere den Plan, gebe das Ergebnis zurück und speichere es in der Datenbank & tbd \\ \hline
	& & & \\ \hline
	PUT & /plans/\{plan\_id\}/proposal & Erstelle und Erhalte einen auf Basis des Plans generierten (neuen) Plan & tbd \\ \hline
	& & & \\ \hline
	GET & /plans/\{plan\_id\}/pdf & Lese PDF-Version des Plans & tbd \\ \hline
	GET & /plans/\{plan\_id\}/share & Lese Link zum teilen des Plans & tbd \\ \hline
	& & & \\ \hline
	GET  & /filters & Lade Filtertypen und Beschränkungen & tbd \\ \hline
	GET & /objective-functions & Lese Liste mit allen vorhandenen Zielfunktionen & tbd \\ \hline
\end{tabularx}

\subsection{Authentifizierung}
\label{subsec:api-auth}
\paragraph{Zugrundeliegende Spezifikation} Die Authentifizierung erfolgt über die Implementierung einer OAuth 2.0 Schnittstelle nach RFC 6749 (Zitieren!).
Für das momentane System ist hierbei nur der \textit{Implicit-Grant-Typ} notwendig, jedoch sollte das System mindestens um den \textit{Authorization-Code-Grant-Typ} erweitert werden können.
\paragraph{Registrierung der Klienten}
Wir unterscheiden zwischen zwei Typen von Klienten: vertrauenswürdigen sowie öffentlichen Klienten. Die Definitionen hierzu finden sich auch in RFC 6749. Bei der Webapp unseres Systems handelt es sich auf Grund der Zugänglichkeit des Codes sowie der Daten um einen öffentlichen Klienten.
\subparagraph{Vertrauenswürdiger Klient}
Für einen vertrauenswürdigen Klienten werden folgende Informationen gespeichert:\\
\begin{tabularx}{\textwidth}{@{} | X | X | @{}}
	\hline
	\textbf{Name} & \textbf{Beschreibung}\\ \hline \hline
	api\_key & Eindeutige öffentliche ID des Klienten. Beginnt mit dem Präfix \enquote{key-} \\ \hline
	api\_secret & Nur dem Klienten bekannte Kennung. Beginnt mit dem Präfix \enquote{secret-} \\ \hline
	scope & Berechtigungen des Klienten. Vorerst immer \enquote{student} \\ \hline
	redirect\_url & URL an welche beim \textit{Implicit-Grant} bzw. \textit{Authorization-Code-Grant} weitergeleitet wird. \\ \hline
	origin & Domains, von welchen aus der Klient auf die Ressourcen zugreifen darf. Wird als Regulärer Ausdruck angegeben. \\
	\hline
\end{tabularx}

\subparagraph{Öffentlicher Klient}
Für einen öffentlichen Klienten werden folgende Informationen gespeichert:\\
\begin{tabularx}{\textwidth}{@{} | X | X | @{}}
	\hline
	\textbf{Name} & \textbf{Beschreibung}\\ \hline \hline
	api\_key & Eindeutige öffentliche ID des Klienten. Beginnt mit dem Präfix \enquote{key-} \\ \hline
	api\_secret & NULL (wird nicht benötigt) \\ \hline
	scope & Berechtigungen des Klienten. Vorerst immer \enquote{student} \\ \hline
	redirect\_url & URL an welche beim \textit{Implicit-Grant} bzw. \textit{Authorization-Code-Grant} weitergeleitet wird. \\ \hline
	origin & Domains, von welchen aus der Klient auf die Ressourcen zugreifen darf. Wird als Regulärer Ausdruck angegeben. \\
	\hline
\end{tabularx}
\paragraph{Schnittstellen}
Für die Authentifizierung werden folgende Schnittstellen verwendet:
\begin{table}
\begin{tabularx}{\textwidth}{@{} | X | X | X | X | @{}}
	\hline
	\textbf{Methode} & \textbf{URL} & \textbf{Beschreibung} & \textbf{Kommunikation} \\ \hline \hline
	GET & /auth/login & Der in RFC 6749, Kapitel 3.1 definierte \textit{Authorization Endpoint} & --- \\ \hline
	POST & /auth/token & Der in RFC 6749, Kapitel 3.2 definierte \textit{Token Endpoint} & --- \\ \hline
\end{tabularx}
\end{table}
Beim Login wird hierbei der Webbrowser des Benutzers an auf die Seite /auth/login weitergeleitet, auf welcher er sich authentifiziert. Anschließend wird er auf die Hauptseite der Webapp weitergeleitet.
\paragraph{Datenaustausch}
Der Datenaustausch beim Aufruf der Schnittstellen ist wie folgt spezifiziert:\\
\begin{tabularx}{\textwidth}{@{} | X | X | X | @{}}
	\hline
	\textbf{Posten} & \textbf{Implicit-Grant} & \textbf{Authorization-Code-Grant} \\ \hline
	Anfrage-Header &
	\begin{itemize}
		\item[]
		\item[client\_id] \{api\_key\}
	\end{itemize} & \\
\end{tabularx}
\subsection{Spezifikation der Statuscodes}
