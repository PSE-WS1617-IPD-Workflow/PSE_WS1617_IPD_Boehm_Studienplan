\FloatBarrier
\subsection{Spezifikation der JSON-Datenklassen}

Im Folgenden werden die zur REST-Kommunikation nötigen JSON-Datenklassen mithilfe der in Kapitel \ref{subsec:rest-notation} Notation spezifiziert. 

\subsubsection*{Modellklassen}

\begin{lstlisting}[language=json]
(*\lbljsonobj{Student}*) = {
	"subject": (*\jsonpart{Studienfach}{id}*),
	"study-start": (*\jsonobj{Studienbeginn}*),
	"passed-modules": (*\jsonlist{\jsonpart{Modul}{id, semester}}*)
}
\end{lstlisting}

\begin{lstlisting}[language=json]
(*\lbljsonobj{Studienbeginn}*) = {
	"semester-type": (*\jsonatom{Semester-Typ}*),
	"year": (*\jsonatom{Jahr}*)
}
\end{lstlisting}

\begin{lstlisting}[language=json]	
(*\lbljsonobj{Studienfach}*) = {
	"id": (*\jsonatom{Studienfach-ID}*),
	"name": (*\jsonatom{Studienfach-Name}*)
}
\end{lstlisting}


\begin{lstlisting}[language=json]
(*\lbljsonobj{Modul}*) = {
	"id": (*\jsonatom{Modul-ID}*),
	"name": (*\jsonatom{Modul-Name}*),
	"category" : (*\jsonatom{Modul-Kategorie}*),
	"semester": (*\jsonatom{Modul-Semester}*),
	"cycle-type": (*\jsonatom{Modul-Turnus}*),
	"creditpoints": (*\jsonatom{Modul-Creditpoints}*),
	"lecturer": (*\jsonatom{Modul-Dozent}*),
	"preference" (*\jsonatom{Modul-Präferenz}*),
	"description": (*\jsonatom{Modul-Beschreibung}*),	
	"constraints": (*\jsonlist{\jsonobj{Constraint}}*)
}
\end{lstlisting}

\begin{lstlisting}[language=json]
(*\lbljsonobj{Constraint}*) = {
	"name": (*\jsonatom{Constraint-Name}*),
	"first": (*\jsonpart{Modul}{id}*),
	"second": (*\jsonpart{Modul}{id}*),
	"type": (*\jsonatom{Constraint-Typ}*),
}
\end{lstlisting}

\begin{lstlisting}[language=json]
(*\lbljsonobj{Filter}*) = {
	"id": (*\jsonatom{Filter-ID}*),
	"name": (*\jsonatom{Filter-Name}*),
	"default-value": (*\jsonatom{Filter-Default}*),
	"tooltip": (*\jsonatom{Filter-Tooltip}*),
	"specification": (*\jsonobj{Filter-Eigenschaften}*)
}
\end{lstlisting}

\begin{lstlisting}[language=json]
(*\lbljsonobj{Filter-Eigenschaften}*) = {
	"type": (*\jsonatom{Filter-Typ}*),
	case type == "range": {
		"min": (*\jsonatom{Filter-Minimum}*),
		"max": (*\jsonatom{Filter-Maximum}*)
	},
	case type == "list": {
		"items": (*\jsonlist{\jsonobj{Filter-Wahlitem}}*)
	},
	case type == "contains": {}
}
\end{lstlisting}
\vspace{-\baselineskip}
Anmerkung: Die case-Notation beschreibt eine Fallunterscheidung; die in den einzelnen Fällen in geschweifte Klammern gekapselten Attribute liegen hierbei in derselben Ebene wie die Fallunterscheidungen. Es versteht sich von selbst, dass stets genau einer der Fälle zutrifft.
\vspace{\baselineskip}

\begin{lstlisting}[language=json]
(*\lbljsonobj{Filter-Wahlitem}*) = {
	"id": (*\jsonatom{Item-ID}*),
	"text": (*\jsonatom{Item-Text}*)
}
\end{lstlisting}

\begin{lstlisting}[language=json]
(*\lbljsonobj{Zielfunktion}*) = {
	"id": (*\jsonatom{Zielfunktion-ID}*),
	"name": (*\jsonatom{Zielfunktion-Name}*),
	"description": (*\jsonatom{Zielfunktion-Beschreibung}*)
}
\end{lstlisting}

\begin{lstlisting}[language=json]
(*\lbljsonobj{Studienplan}*) = {
    "id": (*\jsonatom{Studienplan-ID}*),
    "status": (*\jsonatom{Studienplan-Status}*),
    "creditpoints-sum": (*\jsonatom{Studienplan-Gesamt-Creditpoints}*),
    "name": (*\jsonatom{Studienplan-Name}*),
    "modules": (*\jsonlist{\jsonpart{Modul}{id, name, semester, creditpoints, lecturer}}*),
    "violations": (*\jsonlist{\jsonobj{Constraint}}*)	
}
\end{lstlisting}

\subsubsection*{Kommunikations-Datenstrukturen}

\begin{json}
(*\lbljsonobj{ModulesResult}*) = {
	"modules": (*\jsonlist{\jsonpart{Modul}{id, name, creditpoints, lecturer}}*)
}	
\end{json}

\begin{json}
(*\lbljsonobj{ModuleResult}*) = {
	"module": (*\jsonpart{Modul}{id, name, category, creditpoints, lecturer, description, constraints}*)
}	
\end{json}

\begin{json}
(*\lbljsonobj{StudentResult}*) = {
	"student": (*\jsonobj{Student}*)
}	
\end{json}

\begin{json}
(*\lbljsonobj{StudentPutRequest}*) = {
	"student": (*\jsonobj{Student}*)
}	
\end{json}

\begin{json}
(*\lbljsonobj{StudentDeleteRequest}*) = {
	"student": (*\jsonpart{Student}{id}*)
}	
\end{json}

\begin{json}
(*\lbljsonobj{PlansGetResult}*) = {
	"plans": (*\jsonlist{\jsonpart{Studienplan}{id, status, creditpoints-sum, name}}*)
}	
\end{json}

\begin{json}
(*\lbljsonobj{PlansPostRequest}*) = {
	"plan": (*\jsonpart{Studienplan}{name}*)
}	
\end{json} 

\begin{json}
(*\lbljsonobj{PlansPostResult}*) = {
	"plan": (*\jsonpart{Studienplan}{id, name}*)
}	
\end{json}

\begin{json}
(*\lbljsonobj{PlanResult}*) = {
	"plan": (*\jsonobj{Studienplan}*)
}	
\end{json}

\begin{json}
(*\lbljsonobj{PlanPutRequest}*) = {
	"plan": (*\jsonobj{Studienplan}*)
}	
\end{json}

\begin{json}
(*\lbljsonobj{PlanPatchPostRequest}*) = {
	"plan": (*\jsonpart{Studienplan}{name}*)
}	
\end{json}

\begin{json}
(*\lbljsonobj{PlanPatchPostResult}*) = {
	"plan": (*\jsonpart{Studienplan}{id, name}*)
}	
\end{json}

\begin{json}
(*\lbljsonobj{PlanModulesResult}*) = {
	"modules": (*\jsonlist{\jsonpart{Modul}{id, name, creditpoints, lecturer, preference}}*)
}	
\end{json}

\begin{json}
(*\lbljsonobj{PlanModuleResult}*) = {
	"module": (*\jsonobj{Modul}*)
}
\end{json}

\begin{json}
(*\lbljsonobj{PlanModulePutRequest}*) = {
	"module": (*\jsonpart{Modul}{id, semester}*)
}
\end{json}

\begin{json}
(*\lbljsonobj{PlanModulePutResult}*) = {
	"module": (*\jsonpart{Modul}{id, semester}*)
}
\end{json}

\begin{json}
(*\lbljsonobj{ModulePreferencePutRequest}*) = {
	"module": (*\jsonpart{Modul}{id, preference}*)
}
\end{json}

\begin{json}
(*\lbljsonobj{ModulePreferencePutResult}*) = {
	"module": (*\jsonpart{Modul}{id, preference}*)
}
\end{json}

\begin{json}
(*\lbljsonobj{PlanVerificationResult}*) = {
	"plan": (*\jsonpart{Studienplan}{id, status, violations}*)
}
\end{json}

\begin{json}
(*\lbljsonobj{PlanProposalResult}*) = {
	"plan": (*\jsonpart{Studienplan}{status, modules, violations}*)
}
\end{json}

\begin{json}
(*\lbljsonobj{FiltersResult}*) = {
	"filters": (*\jsonlist{\jsonobj{Filter}}*)
}
\end{json}

\begin{json}
(*\lbljsonobj{ObjectiveFunctionsResult}*) = {
	"functions": (*\jsonlist{\jsonobj{Zielfunktion}}*)
}
\end{json}

\begin{json}
(*\lbljsonobj{DisciplinesResult}*) = {
	"disciplines": (*\jsonlist{\jsonobj{Studienfach}}*)
}
\end{json

\begin{json}
(*\lbljsonobj{SubjectsResult}*) = {
	"subjects": (*\jsonlist{TODO}*)
}
\end{json}