% a few distance declarations for the sequence diagrams
\newcommand{\umlstddt}{3.5}
\newcommand{\umlfragmentdt}{7}
\newcommand{\umlstdafterfragmentdt}{5}
\newcommand{\umltwolinefragmentdt}{9.2}
\newcommand{\umltwolinealtdt}{10}
\newcommand{\umltwolinestddt}{7.2}
%%%%%%%%%%%%%%%%%%%%%%%%%%%%%%%%%%%%%%%%%%%%%%%%%%%%%%%%

\FloatBarrier
\subsection{Beispiele für REST-Kommunikation}

Im folgenden Sequenzdiagramm wird beispielhaft die Umsetzung eines Anwendungsfalls mit Fokus auf die REST-Kommunikation gezeigt (siehe Kapitel \ref{sec:rest}). GET-Parameter sind mit einem Stern (*) versehen und werden in Kapitel \ref{subsec:rest_commands} definiert.

Abbildung \ref{seq:proc-rest-A230} beschreibt die REST-Kommunikations-Abfolge, welche für die Umsetzung des Anwendungsfalls „A230: Studienplan vervollständigen lassen“ notwendig ist. Hierbei wird angenommen, dass ein Vorschlag erfolgreich generiert werden kann und anschließend nicht verworfen wird.

\begin{figure}[H]
	\begin{tikzpicture}  
	\begin{umlseqdiag}  
		\umlbasicobject{Client}
		\umlbasicobject[x=13]{Server}
		% Objective Fns
		\begin{umlcall}[op={GET /objective-functions}, return=\jsonobj{ObjectiveFunctionsResult}]{Client}{Server}  
		\end{umlcall}
		% Module preferences
		\begin{umlcall}[op={GET /filters}, return=\jsonobj{FiltersResult}, dt=\umlstddt]{Client}{Server}  
		\end{umlcall}
		\begin{umlcall}[op={
			GET /plans/\jsonatom{Plan-ID}/modules?filters=    % sic!
			}, return=\jsonobj{PlanModulesResult}, dt=\umlstddt]{Client}{Server}  
		\end{umlcall}
		\begin{umlfragment}[type=opt]
			\begin{umlcall}[op={
				GET /plans/\jsonatom{Plan-ID}/modules/\refgetparams{Modules-Parameter}
			}, return=\jsonobj{PlanModulesResult}, dt=\umlfragmentdt]{Client}{Server}  
			\end{umlcall}
		\end{umlfragment}
		\begin{umlfragment}[type=opt]
			\begin{umlcall}[op={
				GET /plans/\jsonatom{Plan-ID}/modules/\jsonatom{Modul-ID}
			}, return=\jsonobj{PlanModuleResult}, dt=\umlfragmentdt]{Client}{Server}  
			\end{umlcall}
		\end{umlfragment}
		\begin{umlfragment}[type=opt]
			\begin{umlcall}[op={  
				\parbox{.65\linewidth}{\centering
					PUT /plans/\jsonatom{Plan-ID}/modules/\jsonatom{Modul-ID}/preference \\ 
					\jsonobj{ModulePreferencePutRequest}
				}
			}, return=\jsonobj{ModulePreferencePutResult}, dt=\umltwolinefragmentdt]{Client}{Server}  
			\end{umlcall}
		\end{umlfragment}
		% Subjects
		\begin{umlcall}[op={GET /subjects}, return=\jsonobj{SubjectsResult}, dt=\umlstdafterfragmentdt]{Client}{Server}  
		\end{umlcall}
		% Proposal
		\begin{umlcall}[op={
			GET /plans/\jsonatom{Plan-ID}/proposal/\jsonatom{Zielfunktion-ID}/\refgetparams{Proposal-Parameter}
		}, return=\jsonobj{PlanProposalResult}, dt=\umlstddt]{Client}{Server}  
		\end{umlcall}
		% Module info
		\begin{umlfragment}[type=opt]
			\begin{umlcall}[op={
				GET /plans/\jsonatom{Plan-ID}/modules/\jsonatom{Modul-ID}
			}, return=\jsonobj{PlanModuleResult}, dt=\umlfragmentdt]{Client}{Server}  
			\end{umlcall}
		\end{umlfragment}
		\begin{umlfragment}[type=alt, label={Unter neuem Namen speichern}, inner ysep=1.3]
			% Create New Plan
			\begin{umlcall}[op={
				\parbox{.65\linewidth}{\centering
					POST /plans/ \\ 
					\jsonobj{PlansPostRequest}
				}
			}, return=\jsonobj{PlansPostResult}, dt=\umltwolinealtdt]{Client}{Server}  
			\end{umlcall}
			% Write result into plan
			\begin{umlcall}[op={
				\parbox{.65\linewidth}{\centering
					PUT /plans/\jsonatom{Plan-ID} \\ 
					\jsonobj{PlanPutRequest}
				}
			}, return=\jsonobj{PlanPutResult}, dt=\umltwolinestddt]{Client}{Server}  
			\end{umlcall}
		\umlfpart[Übernehmen]
			% Write result into plan
			\begin{umlcall}[op={
				\parbox{.65\linewidth}{\centering
					PUT /plans/\jsonatom{Plan-ID} \\ 
					\jsonobj{PlanPutRequest}
				}
			}, return=\jsonobj{PlanPutResult}, dt=\umltwolinestddt]{Client}{Server}  
			\end{umlcall}
		\end{umlfragment}
	\end{umlseqdiag}
\end{tikzpicture}

\begin{comment}
	\begin{umlseqdiag}  
	\begin{umlfragment}[type={\textbf{sd Test}}, inner ysep=6, inner xsep=2.5]
	%\umlobject[class=A]{a}  
	\umlbasicobject{Client}
	\umlbasicobject[x=9]{Server}
	
	\begin{umlcall}
	[op={GET /bla/blupp/?filter=aaa\&nothing=bla},
	return={\jsonobj{PlanModulesResult}}]
	{Client}{Server}  
	\begin{umlfragment}[type=alt, label=i>5, inner xsep=5]
	\begin{umlcall}[op=sth, with return]{Server}{Server}  
	\end{umlcall}
	\umlfpart[default]  %to define fragment regions
	\begin{umlcall}[op=sthelse, with return]{Server}{Server}  
	\end{umlcall}
	\end{umlfragment}
	\umlcreatecall[class=Class, x=12]{Server}{c} 
	\begin{umlcallself}[type=asynchron]{Server}  
	\end{umlcallself}  
	\end{umlcall} 
	\end{umlfragment}
	\end{umlseqdiag}
\end{comment}
	\caption{Für den Anwendungsfall „A230: Studienplan vervollständigen lassen“ nötige REST-Kommunikations-Abfolge.}
	\label{seq:proc-rest-A230}
\end{figure}