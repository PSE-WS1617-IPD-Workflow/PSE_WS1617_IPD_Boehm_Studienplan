\newpage
\tikzstyle{num}=[fill = green!10]
\tikzstyle{enum}=[fill = yellow!10]
\tikzstyle{bool}=[fill = blue!10]

\section{Datenbankbeschreibung}
Die Produktdaten werden in zwei getrennten Datenbanken gespeichert, eine Datenbank für Moduldaten und eine für Nutzerdaten. Dies ermöglicht es, in Zukunft eine andere Schnittstelle zum Modulhandbuch zu verwenden, wie beispielsweise eine Anbindung an das Vorlesungsverzeichnis. \\
Im Folgenden wird deshalb der Aufbau beider Datenbanken getrennt beschrieben. Zusätzliche Informationen, die nicht aus den Entity-Relationship-Diagrammen hervorgehen, sind in Form von Kommentaren dargestellt. Die Datentypen der Attribute gehen aus folgender Farbkodierung hervor: \\
\begin{tabular}{ |c | c | c | c |}
	\hline
	\tikz{\node[attribute, num]{Integer-Ganzzahl}} &
	\tikz{\node[attribute]{Zeichenkette}} &
	\tikz{\node[attribute, bool]{Boolean-Wert}} &
	\tikz{\node[attribute, enum]{Enum}} \\ \hline
\end{tabular}

\subsection{Moduldaten}
\resizebox{\textwidth}{!} {
\begin{tikzpicture}
	\node[entity] (module) {Module};
	\node[attribute](moduleId)[above left = of module]{\key{module\_id}} edge (module);
	\node[attribute][above = of module]{name} edge (module);
	\node[attribute](obligatory)[above right = of module]{obligatory} edge (module);
	\node[attribute](description)[right = of module]{description} edge (module);
	\node[attribute](creditpoints)[below = of description]{credit\_points} edge (module);	
	\node[attribute](cycle)[below = of creditpoints]{cycle} edge (module);
	
	\node[relationship] (hasaLecturer) [left = of module] {has a} edge node[auto,swap] {0..*}(module);
	\node[entity] (lecturer) [left = 6cm of module]{Lecturer} edge node[auto,swap] {1} (hasaLecturer);	
	\node[attribute](lecturerid)[left = 6cm of moduleId] {\key{lecturer\_id}} edge (lecturer.west);
	\node[attribute](lecturername)[below = 0.5cm of lecturerid]{name} edge (lecturer.west);	
	
	\node[relationship] (hasaKind) [below = 0.5cm of hasaLecturer] {has a} edge node[auto,swap] {0..*}(module);
	\node[entity] (kind) [below = of lecturer]{Kind} edge node[auto,swap] {1} (hasaKind);
	\node[attribute](kindid)[below = of lecturername]{\key{kind\_id}} edge (kind.west);
	\node[attribute](kindname)[below =  0.5cm of kindid]{name} edge (kind.west);	

	\node[relationship] (hasaCategory) [below = 0.5cm of hasaKind] {has a} edge node[auto,swap] {0..*}(module);
	\node[entity] (category) [below = of kind]{Category} edge node[auto,swap] {1} (hasaCategory);	
	\node[attribute](categoryid)[below = of kindname]{\key{category\_id}} edge (category.west);
	\node[attribute](categoryname)[below =  0.5cm of categoryid]{name} edge (category.west);

	\node[relationship] (hasaDiscipline) [below = 0.5cm of hasaCategory] {has a} edge node[auto,swap] {0..*}(module);
	\node[entity] (discipline) [below = of category]{Discipline} edge node[auto,swap] {1} (hasaDiscipline);	
	\node[attribute](disciplineid)[below = of categoryname]{\key{discipline\_id}} edge (discipline.west);
	\node[attribute](disciplinename)[below =  0.5cm of disciplineid]{name} edge (discipline.west);	

	\node[relationship](belrel) [below = 6cm of module] {belongs to} edge node[auto,swap] {2} (module);
	\node[entity](relation)[below = 4cm of belrel]{Relation} edge node[auto,swap] {1} (belrel);	
	\node[attribute](relationid)[below = of relation]{\key{relation\_id}} edge (relation);

	\node[relationship](hasatype) [left = 2cm of relation] {has a} edge node[auto,swap] {0..*} (relation);
	\node[entity](type)[left = 2cm of hasatype]{Type} edge node[auto,swap] {1} (hasatype);	
	\node[attribute](typeid)[below left= of type]{\key{type\_id}} edge (type);
	\node[attribute](typedes)[below right= of type]{Description} edge (type);
\end{tikzpicture}
}
\subsubsection{Enumwerte}
\begin{tabular}{|l|l|}
	\hline
	\textbf{Tabellenspalte} & \textbf{mögliche Werte} \\ \hline
	\texttt{cycle\_type} & winter\_term, summer\_term, both \\ \hline
\end{tabular}
\subsection{Benutzerdaten}
\resizebox{\textwidth}{!} {
\begin{tikzpicture}
	\node[entity] (user) {user};
	\node[attribute][left =of user]{\key{user\_id}} edge (user);
	\node[attribute][below right= of user]{name} edge (user);
	\node[attribute](disc)[below = of user]{discipline\_id} edge (user);
	\node[draw][below right= of disc, text width = 3cm, dashed] {Referenz zum Studiengang in Moduldatenbank} edge [dashed] (disc);
	\node[attribute](start)[below left= of user]{study\_start} edge (user);
	\node[attribute][left= 0.5cm of start]{year} edge (start);
	\node[attribute][below= 0.5cm of start]{semester\_type} edge (start);

	\node[relationship](bel) [above = 2cmof user] {belongs to} edge node[auto,swap] {1} (user);
	
	\node[entity] (plan) [above = 6cm of user]{plan} edge node[auto,swap] {0..*} (bel);
	\node[attribute][above left =of plan]{\key{plan\_id}} edge (plan);
	\node[attribute][left= of plan]{name} edge (plan);
	%\node[attribute][left= of plan]{unique\_id} edge (plan);
	%\node[attribute][below right = of plan]{credit\_points} edge (plan);
	\node[attribute][below left= of plan]{state} edge (plan);
	
	\node[relationship](con) [right =2cm of plan] {contains} edge node[auto,swap] {1} (plan);
	
	\node[relationship](con2) [above = 2cm of plan] {contains} edge node[auto,swap] {1} (plan);
	
	\node[entity] (pref) [above = 6cm of plan] {module\_preference} edge  node[auto,swap] {0..*}(con2);
	\node[attribute](modId2)[right = of pref]{module\_id} edge (pref);
	\node[attribute][left = of pref]{preference\_type} edge (pref);
	\node[attribute][below left = of pref]{\key{preference\_id}} edge (pref);
	
	\node[entity] (entry) [right = 6cm of plan]{module\_entry} edge  node[auto,swap] {0..*}(con);
	\node[attribute][below =of entry]{\key{entry\_id}} edge (entry);
	\node[attribute][above = of entry]{semester} edge (entry);
	\node[attribute](modId)[above right= of entry]{module\_id} edge (entry);
	\node[draw][above left = of modId, text width = 3cm, dashed] {Referenz zum ensprechenden Modul in Moduldatenbank} edge [dashed] (modId) edge [dashed] (modId2);
	
	
	\node[relationship](pass) [above right = 3cm of user] {passed} edge node[auto,swap] {1} (user);
	\draw (pass) edge node[auto,swap] {0..*} (entry);


\end{tikzpicture}
}
\subsubsection{Enumwerte}
\begin{tabular}{|l|l|}
	\hline
	\textbf{Tabellenspalte} & \textbf{mögliche Werte} \\ \hline
	\texttt{preference\_type} & positive, negative \\ \hline
	\texttt{state} & not\_verified, valid, invalid \\ \hline
	\texttt{semester\_type} & winter\_term, summer\_term \\ \hline
\end{tabular}