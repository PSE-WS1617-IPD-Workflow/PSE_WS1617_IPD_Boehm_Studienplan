In der gesamten Client-Implementierung wurden Attributnamen den JSON-Definitionen angepasst, um die Standard-parse- und toJSON-Methoden nutzen zu k�nnen und �bersichtlichkeit durch Einheitlichkeit herzustellen.
\subsubsection{Model-Paket}
\paragraph{Paket modules.*} 
\begin{itemize} [nosep]
\item In der Klasse ModuleCollection wurde die Methode containsModule(moduleId) eingef�gt, um testen zu k�nnen, ob ein Modul bereits in einer Modulcollection enthalten ist. 
\item Der Klasse Preferences wurde die Methode onChange angef�gt, um auf �nderungen der Preferenz reagieren zu k�nnen.
\end  {itemize}
\paragraph{Paket plans.*} In der Klasse Plan wurden einige Methoden hinzugef�gt:  retrieveProposedPlan() um generierten Plan zu erhalten, getEctsSum() und  addModule() um einer SemesterCollection ein neues Modul hinzuzuf�gen. Auf loadVerification() wurde hingegen verzichtet, da diese Information permanent im zugeh�rigen VerificationRecult gespeichert ist, und bei �nderung eine neuer Speicherung und neues endering ausgel��t wird.

\paragraph{System-Paket}
\begin{itemize}
\item Klasse Field wurde eingef�gt, um eine Option, die der Nutzer zur besseren Generierung ausw�hlen kann, zu repr�sentieren.
\item Klasse FieldCollection, welche alle existenten Optionen zur Generierung enth�lt, wurde eingef�gt.
\end{itemize}

\paragraph{User-Paket}

\subsubsection{Router-Paket}

\subsubsection{Storage-Paket}
\subsubsection{View-Paket}

\paragraph{Filter-Paket}

\paragraph{Uielement-Paket}

\paragraph{Uipanel-Paket} 
\begin{itemize}
\item In GenerationWizardComponent2 und in SignUpWizardComponent2 werden die onChange()-Methoden nicht gebraucht, da die fertigen Seiten moduleFinder beziehungsweise ProfilPage nutzen, und diese selbstst�ndig triggern und speichern. 
\item GenerationWizardComponent3 nutzt die eingef�hrte FieldCollection (siehe 2.7.1) und ben�tigt neben onChange() zwei weitere Methoden um auf das Bet�tigen der Slider zu reagieren.
\item Die Klasse SignUpWizardComponent1 ben�tigt eine weitere onChange-Methode, da sie zwei Events enth�lt und eine Methode beginning() zur gekapselten Erstellung der Studienstart-Auswahl-Daten.
\end{itemize}

\paragraph{subview}
\begin{itemize}
\item NotFoundPage eingef�gt
\end{itemize}



