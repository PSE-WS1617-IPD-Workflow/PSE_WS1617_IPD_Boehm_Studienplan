\section{Server einrichten}
\subsection{Datenbank einrichten}
\begin{itemize}
	\item beide Datenbanken aus den Dumps \texttt{moduledata\_dump.sql} und \texttt{userdata\_dump.sql} mit MySQL erzeugen
	\item in der Nutzer-Datenbank folgendes Statement ausführen \\
	\begin{lstlisting}[language=SQL, tabsize=2]
INSERT INTO `rest_client` 
VALUES (1, 'key-26hg02lsa',
		'secret-jg921tjg0', '.*',
		'http://localhost/processLogin');
	\end{lstlisting}
	\item unter \texttt{studyplan\_server/src/main/resources} die Dateien \texttt{moduledata.cfg.xml} und \texttt{userdata.cfg.xml} bearbeiten und in Zeile 11 die Datenbankverbindung anpassen (connection.url, connection.username und connection.password müssen geändert werden)
	\begin{lstlisting}[language=XML, tabsize=2, frame = single, caption={Auszug aus *data.cfg.xml}, captionpos=b]
<!-- Database connection settings -->
<property name="connection.driver_class">
	com.mysql.jdbc.Driver
</property>
<property name="connection.url">
	jdbc:mysql://path/to/database
</property>
<property name="connection.username">username</property>
<property name="connection.password">password</property>
	\end{lstlisting}
\end{itemize}
\subsection{Tomcat einrichten}
\begin{itemize}
	\item \textit{Apache Tomcat 8.5} herunterladen oder direkt aus der ZIP-Datei entpacken (im letzteren Fall ist Tomcat bereits konfiguriert)
	\item ansonsten müssen folgende im Ordner \texttt{conf} von Tomcat angepasst werden, hier kann sich an den Konfigurationen der fertig gepackten Variante orientiert werden
	\begin{itemize}
		\item \texttt{context.xml}: bei \texttt{<context>} sollte \texttt{reloadable} auf \texttt{true} gesetzt werden (dies ermöglicht einen Redeploy ohne Neustart des Servers)
		\item   \texttt{server.xml}: hier kann in Zeile 69 der Port eingestellt werden auf dem Tomcat läuft, in unserem Fall 9999
		\item \texttt{tomcat-users.xml}: Hier muss die Role \texttt{student} und entsprechende Nutzer die dieser angehören hinzugefügt werden. Nur Nutzer die hier hinterlegt sind, können sich auf dem System einloggen. \\
		Dies ist das Äquivalent zu einem auf dem Shibboleth Identity Provider registrierten Nutzer, er ist berechtigt unsere Anwendung zu nutzen, hat aber zunächst kein Benutzerkonto.
	\end{itemize}
	\item Wurde die fertig konfigurierte Installation verwendet, so sind bereits die Nutzer \textit{max\_mustermann} und \textit{peter\_schmidt} hinterlegt, beide mit dem Passwort \textit{123}.
\end{itemize}
\subsection{Server kompilieren und ausführen}
\begin{itemize}
	\item Voraussetzung ist eine korrekt eingerichtete Installation von \textit{Apache Maven}
	\item Über die Kommandozeile in das Verzeichnis \texttt{studplan\_server} navigieren
	\item \texttt{mvn package} ausführen
	\item im Verzeichnis \texttt{target} findet sich die Datei \texttt{studyplan.war}. Diese muss nun in das \texttt{webapp} Verzeichnis der Tomcat Installation kopiert werden.
	\item der Server kann nun über das Skript \texttt{startup.sh} (bzw. \texttt{*.bat}) im \texttt{bin} Verzeichnis der Tomcat-Installation gestartet werden
	\item der Studyplan Applikationsserver läuft nun unter \texttt{http://localhost:9999/studyplan}
	\item angehalten wird Tomcat über das \texttt{shutdown} Skript im bin Verzeichnis
	\item Alternativ kann der Server auch ohne lokale Tomcat-Installation über das Tomcat-Maven-Plugin gestartet werden. Hierfür muss im Verzeichnis \texttt{studyplan\_server} \texttt{mvn tomcat7:run} ausgeführt werden
\end{itemize}
