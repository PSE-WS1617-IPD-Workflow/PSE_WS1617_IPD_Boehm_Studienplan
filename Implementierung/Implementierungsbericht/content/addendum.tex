\section{Bedienungsanleitung}

StudyPlaner ist eine WebApp die dafür sorgt dass ein Benutzer sein Studium einfach, schnell und sicher planen kann. Dafür muss er sich registrieren und ein Paar Informationen bezüglich sein Studium (Studienfach, Studienbeginn, bestandene Modulen..) eingeben, damit die richtigen informationen für ihn geladen werden können. \\

Nach der Anmeldung wird er auf die Hauptseite weitergeleitet dort kann er seine Plänen verwalten ( erstellen, duplizieren, als PDF exportieren und löschen). \\

Will der Benutzer einen Plan erstellen, so drückt er auf den Erstellungsbutton und gibt einen Namen ein. Die Planansicht-Seite ist eine übersichtliche Oberfläche mit einem Plan auf der linken Seite und einer Modulen-Liste auf der rechten.\\

Die bereits bestandene Module bzw. vergangene Semestern sind schon im Plan eingebettet bzw. ausgefüllt.\\
So kann der Benutzer neue Semestern einfügen und Modulen mit Drag and Drop aus der Modulen-Liste in den Semestern des Plans fallen lassen. Die ECTS anzahl der Semestern und des ganzen Plans werden somit aktualisiert. \\

Die Modulen-Liste stellt verschiedene Filtern zur Verfügung, damit es einfacher wird einen Modul mittels verschiedenen kriterien gefunden werden, sowie Name, angebotenes Semester,  Veranstaltungsart, Fachrichtung, Kategorie, ECTS-Bereich und ob der Modul pflichtig ist.\\

Der Benutzer hat auch die Möglichkeit die Modulen positiv oder negativ zu bewerten (die Bewertungen werden auch bei der Generierung berücksichtigt).\\
Klickt er auf den Verifybutton wird der derzeitige Plan anhand der Modulen-Constraints und der studienfachspeziefische-Constraints verifiziert. Wenn Constraint-Verletzungen gefunden werden, werden in einem Pop-up Dialog Fenster angezeigt werden und die fehlerhaften Module rot umrahmt. Damit kann der Benutzer den Plan dementsprechend anpassen.\\

Der Vervollständigen-button dient zur vervollständigung des bereits existierenden Plans mit angemessenen Modulen. Klickt der Benutzer darauf  wird er auf den Generierungsansicht weitergeleitet. Hier kann er das Maximum ECTS Punkte Per Semester sowie die Zielfunktionen auswählen. Eine Zielfunktion ist ein Kriterium anhand dessen einen Plan optimisiert wird. Wählt der Benutzer die mindest Anzahl an Semestern Zielfunktion, wird der Plan so vervollständigt, dass er möglichst weniger Semestern enthält.  \\

Wenn aus dem bereits existierenden Plan einen neuen vervollständigt Plan anhand der eingegebenen Daten generiert werden kann. Wird den neuen Plan angezeigt und mit einer neuen Leiste, die den Plan zu übernehmen, zu verwerfen, oder unter einen neuen Namen zu speichern anbietet. Der Benutzer muss nur noch wählen welche Aktion zu unternehmen.\\

Der Benutzer kann jederzeit seinen Profil bearbeiten, abmelden und mittels die bei der Registrierung eingegebenen Login-Daten sich wieder anmelden.
