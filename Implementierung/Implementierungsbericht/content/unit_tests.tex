\section{Testfälle}
In der Implementierungsphase wurden bereits erste Integrationstest durchgeführt. So wurden fast alle möglichen Nutzeraktionen einmal durchgespielt, um eine eine grundlegende Qualität der Implementierung zu gewährleisten
\subsection{Server}
Auf dem Server wurden noch nicht für alle Klassen Unittests erstellt, da es sich hauptsächlich um Klassen zur Darstellung von REST-Schnittstellen und simple Java-Objekte handelt, die nur Getter und Setter besitzen. Die tatsächliche Funktionalität kann erst in Verbindung mit einer Datenbank getestet werden. Diese Integrationstests werden in Qualitätssicherungsphase stattfinden.


Unittests:


\begin{tabular}{|>{\ttfamily} l | p{0.6\textwidth}|}
	\hline
	\normalfont{\textbf{Testklasse}} & \textbf{Beschreibung} \\
	\hline
	CategoryTest & Getter und Setter für \texttt{Category} getestet \\
	\hline
	DisciplineTest & Getter und Setter für \texttt{Discipline} getestet \\
	\hline
	SemesterTest & Test der Semester-Abstandberechnung zwischen:
	\begin{itemize}
		\item zwei Wintersemestern
		\item zwei Sommersemestern
		\item zwischen Winter- und Sommersemester
		\item zwischen Sommer- und Wintersemester
	\end{itemize}
	Test der compareTo-Methoden \\
	\hline
	StandardVerifierTest & Testet ob der \texttt{StandardVerifier} folgendes erkennt:
	\begin{itemize}
		\item fehlende Pflichtmodule
		\item Verletzung von Rule-Group-Bedingungen
		\item Verletzung von Bereichsbedingungen (Fields)
	\end{itemize} \\
	\hline
	StandardVerifierConstraintTest & Testet ob der die Verletzung folgender \texttt{ConstraintTypes} in unterschiedlichen Ausartungen erkennt:
	\begin{itemize}
		\item Verletzung von Überlappung 
		\item Verletzung von Plan-Zusammengehörigkeit 
		\item Verletzung von Semester-Zusammengehörigkeit 
		\item Verletzung von Voraussetzungen
	\end{itemize}
	Dadurch wurden auch die \texttt{isValid}-Methoden der entsprechenden \texttt{ConstraintTypes} getestet \\
	\hline

\end{tabular}