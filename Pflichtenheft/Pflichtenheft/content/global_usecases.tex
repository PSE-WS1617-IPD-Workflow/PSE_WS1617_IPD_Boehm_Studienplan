\section{Globale Testfälle}

\renewcommand{\arraystretch}{1.24}  % for proper row spacing
\setlength{\LTpre}{3pt}  % adjust top margin of longtables
\setlength{\LTpost}{0pt} % adjust bottom margin of longtables

% scenario environment
% #1: Scenario title
\newenvironment{scenario}[1]{
	\vspace{-\baselineskip}
	\subsubsection{#1} 
	\vspace{-\baselineskip}
	\begin{enumerate}[nosep]
}{
	\end{enumerate}
}

% scenario* environment
% #1: Scenario title
% #2: Initial description
\newenvironment{scenario*}[2]{
	\vspace{-\baselineskip}
	\subsubsection{#1} \vspace{-\baselineskip}
	\hspace{0pt}#2  \vspace{-\baselineskip}
	\begin{enumerate}[nosep]
}{
	\end{enumerate}
}


% usecase environment
% #1: Title of the use case, like "\lA{n}: Title"
% #2: Initial state description
\newenvironment{usecase}[2]{
	\subsubsection*{#1}  
	\addcontentsline{toc}{subsubsection}{#1} 
	\vspace{-\baselineskip}\textbf{Ausgangs-Stand: } #2
	\begin{longtable}{|p{.44\linewidth}|p{.55\linewidth}|}
		\hhline{|=|=|}
		\textbf{Aktion} & \textbf{Reaktion} \\
		\hline 
		\endfirsthead
		
		\hline
		\textbf{Aktion} & \textbf{Reaktion} \\
		\endhead
		
		\hhline{|=|=|}
		\endlastfoot
}{
	\end{longtable} \vspace{-12pt}
}

% tblitemize environment
% Provides itemize optimized for table cells
\newenvironment{tblitemize}{
	\begin{itemize}[nosep,leftmargin=12pt]
}{
	\end{itemize}\hspace{0pt}\vspace{-\baselineskip}
}

% lA command (label A...)
% declares a usecase label
% #1: number of the A-abbrev. (without leading A)
\newcommand{\lA}[1]{\label{A#1}A#1}

% refA command (refer to A...)
% creates a linked reference to the given usecase
% #1: number of the A-abbrev. (without leading A)
\newcommand{\refA}[1]{\hyperref[A#1]{/A#1/}}


\subsection{Testszenarien}
Folgende Funktionssequenzen müssen überprüft werden:

\begin{scenario}{Erststart mit „halbherziger Bedienung“}
	\item \nameref{A10} (ohne Angabe bereits belegter Module)
	\item \nameref{A50}
	\item \nameref{A220} – Ergebnis „fehlerhaft“ (da unvollständig)
	\item \nameref{A230} mit anschließendem Verwerfen
	\item \nameref{A240}
	\item \nameref{A80} 
	\item \nameref{A30}
\end{scenario}

\begin{scenario}{Einfache Vervollständigung}
	\item \nameref{A20}
	\item \nameref{A40} – erste zwei Semester anschließend belegt
	\item \nameref{A50}
	\item \nameref{A230} mit anschließendem Übernehmen des Plans
	\item \nameref{A60} 
	\item \nameref{A240}
	\item \nameref{A85}
	\item \nameref{A30}
\end{scenario}

\begin{scenario*}{Bearbeitung eines Studienplans}
	{Nutzer ist bereits eingeloggt und hat mind. einen Studienplan.}
	\item \nameref{A55}
	\item \nameref{A110} 
	\item \nameref{A130} und wieder schließen
	\item \nameref{A140}
	\item \nameref{A160}
	\item \nameref{A110} 
	\item \nameref{A150} und über das „Anker“"=Symbol eines Bestandteils davon hovern
	\item \nameref{A215} (danach: ausgeblendet)
	\item \nameref{A170}
	\item \nameref{A110}
	\item \nameref{A190}
	\item \nameref{A180} (selbes Modul – es ist dann im Like-Status) 
	\item \nameref{A140}
	\item \nameref{A120}
	\item \nameref{A240}
\end{scenario*}

\begin{scenario}{Profil bearbeiten}
	\item \nameref{A55}
	\item \nameref{A215} (danach: eingeblendet)
	\item \nameref{A110}
	\item \nameref{A40} – dabei Änderung der Semester-Belegung
	\item Anschließend sollte die Änderung im Plan gezeigt werden und die Suchleiste sich im selben Zustand befinden wie vor Schritt 4.
\end{scenario}

\begin{scenario}{Vervollständigung mit mehreren Alternativen}
	\item \nameref{A20}
	\item \nameref{A50}
	\item \nameref{A110}
	\item \nameref{A140} oder \nameref{A150}
	\item Schritte 3–4 mehrmals wiederholen, sodass Abhängigkeitsfehler vorhanden sind und der Plan noch unvollständig ist
	\item \nameref{A220}: Es werden Abhängigkeitsfehler gemeldet
	\item Mittels \nameref{A160} und \nameref{A170} Abhängigkeitsfehler beheben
	\item \nameref{A220}: Der Plan ist unvollständig
	\item \nameref{A230} mit anschließendem Speichern unter neuem Namen
	\item \nameref{A240}
	\item \nameref{A55} (den in Schritt 2 erstellten Plan)
	\item \nameref{A230} mit anderen Zielkriterien als in Schritt 9 und anschließendem Speichern unter neuem Namen
	\item \nameref{A240}
	\item \nameref{A100} (die zwei generierten Pläne)
\end{scenario}

\begin{scenario}{Studienpläne duplizieren und löschen}
	\item \nameref{A20}
	\item \nameref{A50}
	\item \nameref{A240}
	\item \nameref{A70}
	\item \nameref{A90}: Duplizieren aller vorhandenen Studienpläne
	\item Mehrmaliges Wiederholen von Schritt 5.
	\item \nameref{A90}: Alle Studienpläne löschen.
\end{scenario}

\begin{scenario}{Semester"=Zeilen anpassen}
	\item \nameref{A20}
	\item \nameref{A50}
	\item Mehrmals \nameref{A210}, bis keine mehr vorhanden sind.
	\item Mehrmals \nameref{A200}
	\item \nameref{A240}
\end{scenario}



\subsection{Anwendungsfälle}

\begin{center}
	\resizebox{\textwidth}{!} {
		\begin{tikzpicture}
	\begin{umlsystem}{Studienplan-Verifizierung}
		\umlusecase[x=-3]{Studienplan anlegen}
		\umlusecase[x=3, y=-1]{Modul hinzufügen}
		\umlusecase[x=3, y=1]{Modul suchen}
		\umlusecase[x=-3, y=-2]{Studienplan verifizieren}
		\umlusecase[x=-3, y=-4]{Konlifkte anzeigen}
	\end{umlsystem}

	\umlactor[x=-8]{Nutzer}

	\umlassoc{Nutzer}{usecase-1}
	\umlassoc{Nutzer}{usecase-3}
	\umlassoc{Nutzer}{usecase-4}
	\umlassoc{Nutzer}{usecase-5}

	\umlinclude{usecase-1}{usecase-2}
	\umlinclude{usecase-2}{usecase-3}
\end{tikzpicture}
	}
\end{center}



\begin{usecase}{\lA{10}: Erstanmeldung}
	{Geöffnete Seite, unangemeldet}
	Nutzer loggt sich zum ersten Mal via Shibboleth Identity Provider ein
	& Dem Nutzer erscheint eine Willkommensseite; mitsamt der Eingabe"=Formulare wie in \refA{40} beschrieben.\\ 
	\hline
	Nutzer füllt die Formulare aus.
	& Dem Nutzer erscheint die Hauptansicht, auf der bislang keine Studienpläne vorhanden sind. 	
\end{usecase}

\begin{usecase}{\lA{20}: Login}
	{Geöffnete Seite, unangemeldet}
	Nutzer loggt sich via Shibboleth Identity Provider ein
	& Dem Nutzer erscheint die Hauptansicht mit ggfs. bereits angelegten Studienplänen
\end{usecase}

\begin{usecase}{\lA{30}: Logout}
	{Geöffnete Seite, angemeldet, beliebige Ansicht}
	Nutzer klickt auf den Logout"=Knopf.
	& Dem Nutzer erscheint die Login-Seite, er wird bis zur nächsten Anmeldung nicht mehr als eingeloggt erkannt.
\end{usecase}

\begin{usecase}{\lA{40}: Profil bearbeiten}
	{Geöffnete Seite, angemeldet, beliebige Ansicht}
	Nutzer klickt auf den Profil"=Knopf.
	& Dem Nutzer erscheint ein Formular zur Eingabe von Studienbeginn und Studiengang. \\ 
	\hline
	Nutzer gibt diese Informationen ein und drückt auf den Weiter"=Knopf.
	& Dem Nutzer erscheint ein Formular zur Eingabe des aktuellen Studienstands: Er kann festlegen, welche Prüfungsleistungen er schon begonnen und in welchem Semester er sie bestanden hat. Sechs leere Semester"=Zeilen sind als Startwert vorgegeben.\\
	\hline
	Nutzer kann nun Module als begonnen markieren, indem er sie in den Studienplan ins jeweilige Semester hineinzieht (\refA{140}), und sie als bestanden markieren, indem er den entsprechenden Knopf im jeweiligen Modul anwählt. Modulfilterung (\refA{110}) ist in dieser Ansicht möglich.
	Nach Eingabe dieser Informationen drückt er den Weiter"=Knopf.
	& Dem Nutzer erscheint die Ansicht, in welcher er den Profil"~Knopf betätigt hat, oder im Fall \refA{10} die Hauptansicht.
\end{usecase}

\begin{usecase}{\lA{50}: Neuen Studienplan anlegen}
	{Hauptansicht}
	Nutzer klickt auf den Knopf „Neuen Studienplan erstellen“.
	& Dem Nutzer erscheint ein Popup, welches ihn nach dem Studienplan-Namen fragt; voreingestellt ist „Neuer~Studienplan~1“. \\
	\hline
	Nutzer gibt gewünschten Namen ein und bestätigt.
	& Der neue Studienplan öffnet sich in der Bearbeitungsansicht und wird zur Liste der bereits erstellten Pläne hinzugefügt. Er gilt als bislang nicht überprüft und enthält bereits die in der Profilansicht hinzugefügten bereits belegten Module.
\end{usecase}

\begin{usecase}{\lA{55}: Studienplan anzeigen}
	{Hauptansicht, es exist. mind. ein Studienplan}
	Nutzer klickt auf den Namen eines Studienplans oder rechts davon auf „Anzeigen“.
	& Der gewählte Studienplan öffnet sich in der Bearbeitungsansicht.
\end{usecase}
	
\begin{usecase}{\lA{60}: Studienplan umbenennen}
	{Bearbeitungsansicht mit offenem Studienplan}
	Nutzer klickt auf den Namen des Studienplans
	& Dem Nutzer erscheint ein Popup, welches ihn nach dem neuen Studienplan"=Namen fragt; voreingestellt ist der alte Name. \\
	\hline
	Nutzer gibt gewünschten Namen ein und bestätigt.
	& Das Popup verschwindet und die vorherige Ansicht erscheint wieder, wobei sich der Name des Studienplans sich geändert hat.
\end{usecase}

\begin{usecase}{\lA{70}: Studienplan duplizieren}
	{Hauptansicht, es exist. mind. ein Studienplan}
	Nutzer klickt neben einem Studienplan „\$Name“ auf „Duplizieren“.
	& Eine Kopie des Studienplans namens „\$Name – Kopie~\#n“ taucht unter dem Kopierten in der Studienplanliste auf (\#n beschreibt die kleinste Zahl $\ge 1$, die keine Namenskollisionen hervorruft).
\end{usecase}

\begin{usecase}{\lA{80}: Studienplan löschen}
	{Hauptansicht, es exist. mind. ein Studienplan}
	Nutzer klickt neben einem Studienplan auf „Löschen“.
	& Nutzer wird mittels Dialog gebeten, das Löschen des Studienplans zu bestätigen. \\
	\hline
	Nutzer entscheidet sich für Bestätigung oder Abbruch.
	& Der Dialog verschwindet, dem Nutzer erscheint die Hauptansicht. Falls er das Löschen bestätigt hat, existiert der genannte Studienplan nun nicht mehr.
\end{usecase}

\begin{usecase}{\lA{85}: Studienplan exportieren}
	{Hauptansicht, es exist. mind. ein Studienplan}
	Nutzer klickt neben einem Studienplan auf „Exportieren“.
	& Das System generiert eine PDF"=Zusammenfassung des Studienplans, welche dem Nutzer vom Browser zum Download angeboten wird.
\end{usecase}

\begin{usecase}{\lA{90}: Mehrere Studienpläne duplizieren/löschen}
	{Hauptansicht, es exist. mind. ein Studienplan}
	Nutzer wählt einen oder mehrere Studienpläne mittels der Anwahlkästchen aus (oder auch alle durch Wählen des obersten Hakens in der Leiste).
	Danach wählt der Nutzer im Aktions"=Wahlfeld „Duplizieren“ oder „Löschen“.
	& Es folgt das Vorgehen wie in \refA{70} bzw. \refA{80}, zusammengefasst angewandt auf die markierten Studienpläne.
\end{usecase}

\begin{usecase}{\lA{100}: Vergleichsansicht für Studienpläne}
	{Hauptansicht, es exist. mind. zwei Studienpläne}
	Nutzer wählt genau zwei Studienpläne mittels Anwahlkästchen aus. Im Aktions"=Wahlfeld betätigt er die Vergleichsansicht.
	& Dem Nutzer erscheint die Vergleichsansicht. \\
	\hline
	Nutzer schließt die Vergleichsansicht. 
	& Der Nutzer kehrt zur Hauptansicht zurück.
\end{usecase}

\begin{usecase}{\lA{110}: Module in der Suchleiste filtern}
	{Bearbeitungsansicht, Studienplan geöffnet, Suchleiste wird angezeigt}
	Der Nutzer kann in der Suchleiste Module filtern durch Wählen...
	\begin{tblitemize}
		\item des ECTS"=Intervalls (Klick auf „ECTS“ und Ziehen an den Reglern)
		\item der Veranstaltungsart $\in\hspace{-3pt}\{$ Vorlesung, Praktikum, Seminar $\}$ (Klick auf „Art“ und Auswahl)
		\item der Kategorie (bzw. des Themenbereiches) (Klick auf „Kategorie“ und Auswahl)
		\item des Turnus $\in \{$ WS, SS, WS/SS $\}$ (Klick auf „WS/SS“ und Auswahl)
		\item ob Pflicht-, Wahlveranstaltungen oder beides anzuzeigen ist (Klick auf „Pflicht/Wahl“ und Auswahl)
		\item der Fachrichtung (Klick auf „Fachrichtung“ und Auswahl)
		\item ob bereits platzierte Module anzuzeigen sind (Klick auf „mit Platzierten?“ und Auswahl)
		\item eines Suchbegriffes, nach welchem die Titel der Module gefiltert werden (Eingabe von Text ins Suchfeld)
	\end{tblitemize}
	& In der Suchleiste werden entspr. der Nutzerfilterung alle Module angezeigt, die...
	\begin{tblitemize}
		\item im gewählten ECTS"=Intervall liegen
		\item der gewählten Veranstaltungsart entsprechen
		\item zur gewählten Kategorie gehören
		\item im gewählten Turnus stattfinden
		\item Pflicht-, Wahlveranstaltungen oder beides sind
		\item zur gewählten Fachrichtung gehören
		\item bereits platziert worden sind oder nicht
		\item den gewählten Suchbegriff im Titel enthalten
	\end{tblitemize} \\
	\hline
	Der Nutzer kann gesetzte Filter durch Klicken auf das Kreuzchen im Filter"=Knopf wieder zurücksetzen.
	& Entsprechende Filter treten außer Kraft und die Suchleiste aktualisiert sich wie oben.
\end{usecase}

\begin{usecase}{\lA{120}: Info-Popup zu einem Modul anzeigen}
	{Bearbeitungsansicht, Studienplan geöffnet, mind. ein Modul im Studienplan verteilt}
	Nutzer klickt in der Tabelle auf ein Modul.
	& Das Modul wird farblich hervorgehoben. \\
	\hline
	Nutzer klickt erneut auf das Modul.
	& Auf der Seite öffnet sich ein Info-Popup über der Tabelle; es enthält
	\begin{tblitemize}
		\item Titel und Modulnummer
		\item Dozent, ECTS, Modulbeschreibung, evtl. Turnus, Dauer (in Semestern)
		\item aktuell gewählte ECTS (bezügl. gewählter Teilleistungen)
		\item optionale Teilleistungen und Pflichtbestandteile (dazu ECTS und Verantwortliche)
		\item Symbol"=Buttons zum Liken (\refA{180}) und Disliken (\refA{190}) des Moduls
	\end{tblitemize} \\
	\hline
	Nutzer kann nun das Modul liken/disliken (\refA{180} bzw. \refA{190}).
	& Reaktion wie in \refA{180} bzw. \refA{190}. \\
	\hline
	Nutzer klickt im Popup auf den Schließen"=Button. 
	& Das Popup schließt sich, Rückkehr zur Bearbeitungsansicht; das Modul bleibt hervorgehoben. \\
	
	(Alternative: Nutzer klickt auf eine Stelle außerhalb des Popups.)
	& (Alternative: Das Popup schließt sich, der Mausklick außerhalb des Popups wird entsprechend ausgeführt.)
\end{usecase}

\begin{usecase}{\lA{130}: Info-Leiste zu einem Modul anzeigen}
	{Bearbeitungsansicht, Studienplan geöffnet, mind. ein Modul in der Suchleiste aufgelistet}
	Nutzer klickt auf ein Modul in der Suchleiste.
	& Die Suchleiste verwandelt sich in eine Info-Leiste und zeigt den Inhalt von \refA{120} an. \\
	\hline
	Nutzer klickt in der Info-Leiste auf den Zurück-Knopf.
	& Die Leiste kehrt zur exakt vorherigen Suchansicht zurück.
\end{usecase}

\begin{usecase}{\lA{140}: Modul in Plan einfügen}
	{Bearbeitungsansicht, Studienplan geöffnet, mind. eine Semester"=Zeile vorhanden, mind. ein unplatziertes Modul oder Teilmodul in der Suchleiste aufgelistet}
	Der Nutzer hält seine Maustaste über einem Modul oder einem Teilmodul in der Suchleiste gedrückt, zieht es in eine Semester"=Zeile und lässt es fallen.
	& Das Modul erscheint an der Zielstelle und gilt als platziert, Gesamt"= und Zeilen"=ECTS erhöhen sich. Der Überprüfungsstatus des Plans ändert sich zu „nicht überprüft“.
\end{usecase}

\begin{usecase}{\lA{150}: Modul mit Teilmodulen in Plan einfügen}
	{Bearbeitungsansicht, Studienplan geöffnet, mind. eine Semester"=Zeile vorhanden, mind. ein unplatziertes Modul mit Teilmodulen in der Suchleiste aufgelistet}
	Der Nutzer hält seine Maustaste über einem Modul mit Teilmodulen in der Suchleiste gedrückt, zieht es in eine Start"=Semester"=Zeile und lässt es fallen.
	& Die Teilmodule erscheinen in der Tabelle in aufeinanderfolgenden Semester"=Zeilen (ggfs. werden neue Semester"=Zeilen hinzugefügt), beginnend bei der Start"=Zeile. Die Teilmodule gelten als platziert, Gesamt"= und Zeilen"=ECTS erhöhen sich. Der Überprüfungsstatus des Plans ändert sich zu „nicht überprüft“. \\
	\hline
	Der Nutzer kann nun mit der Maus über das „Anker“"=Symbol eines Teilmoduls hovern.
	& Alle zum selben Modul gehörigen Teilmodule werden als zusammenhängend hervorgehoben, solange sich die Maus über dem Symbol befindet.
\end{usecase}

\begin{usecase}{\lA{160}: Modul aus Plan löschen}
	{Bearbeitungsansicht, Studienplan geöffnet, mind. ein Modul im Studienplan verteilt}
	Der Nutzer klickt auf ein Modul in der Tabelle.
	& Das Modul erscheint ausgewählt. \\
	\hline
	Der Nutzer klickt auf den Löschen-Knopf des Moduls. 
	& Das Modul verschwindet und gilt als niht platziert, die ECTS der entsprechenden Semester"=Zeile und die Gesamt"=ECTS verringern sich. Der Überprüfungsstatus des Plans ändert sich zu „nicht überprüft“.
\end{usecase}

\begin{usecase}{\lA{170}: Modul innerhalb Plan verschieben}
	{Bearbeitungsansicht, Studienplan geöffnet, mind. ein Modul im Studienplan verteilt, mind. zwei Semester"=Zeilen in der Tabelle}
	Der Nutzer hält seine Maustaste über einem Modul in der Tabelle gedrückt, zieht es in eine Semester"=Zeile ungleich der vorherigen und lässt es fallen.
	& Das Modul verschiebt sich dorthin, die Semester-ECTS der Ausgangs"= und der Ziel"=Zeile ändern sich entsprechend und der Überprüfungsstatus de Studienplans ändert sich zu „nicht überprüft“.
\end{usecase}

\begin{usecase}{\lA{180}: Modul liken}
	{Bearbeitungsansicht, Studienplan geöffnet, mind. ein Modul in der Suchleiste aufgelistet}
	Nutzer klickt auf das Like-Symbol eines der in der Suchleiste aufgeführten Module.
	& Modul bereits im Like-Zustand? $\Rightarrow$ Modul verlässt den Like-Zustand \newline
	Ansonsten $\Rightarrow$ verlässt das Modul evtl. den Dislike-Zustand und kommt in den Like-Zustand. \newline
	Das Like-Symbol erscheint entspr. aktiv bzw. inaktiv. \\
\end{usecase}

\begin{usecase}{\lA{190}: Modul disliken}
	{Bearbeitungsansicht, Studienplan geöffnet, mind. ein Modul in der Suchleiste aufgelistet}
	Nutzer klickt auf das Dislike-Symbol eines der in der Suchleiste aufgeführten Module.
	& Modul bereits im Dislike-Zustand? $\Rightarrow$ Modul verlässt den Dislike-Zustand \newline
	Ansonsten $\Rightarrow$ verlässt das Modul evtl. den Like-Zustand und kommt in den Dislike-Zustand. \newline
	Das Dislike-Symbol erscheint entspr. aktiv bzw. inaktiv. \\
\end{usecase}

\begin{usecase}{\lA{200}: Semester im Plan hinzufügen}
	{Bearbeitungsansicht, Studienplan geöffnet}
	Der Nutzer klickt in der Semester"=Leiste auf „Weiteres Semester hinzufügen“.
	& In der Tabelle erscheint unten eine neue leere Semester"=Zeile mit 0~ECTS.
\end{usecase}

\begin{usecase}{\lA{210}: Semester aus Plan löschen}
	{Bearbeitungsansicht, Studienplan geöffnet, mind. eine Semester"=Zeile in der Tabelle vorhanden}
	Der Nutzer klickt in einer Semester"=Zeile auf „Semester löschen“.
	& Falls die Zeile nicht leer ist, wird der Nutzer mittels Popup gebeten, das Löschen der Zeile zu bestätigen. Falls doch, entfällt das Popup und es erfolgt sofort die nächste Reaktion.\\
	\hline
	Der Nutzer entscheidet sich für Bestätigen oder Abbruch.
	& Der Nutzer kehrt zur vorherigen Bearbeitungsansicht zurück. Falls er das Löschen bestätigt hat, verschwindet die Semester"=Zeile; dadurch gelten alle darin enthaltenen Module nicht mehr als platziert und der Überprüfungsstatus des Studienplans ändert sich zu „nicht überprüft“; ferner aktualisieren sich die Gesamt-ECTS.
\end{usecase}

\begin{usecase}{\lA{215}: Abgeschlossene Semester im Plan ein"=/ausblenden}
	{Bearbeitungsansicht, Studienplan geöffnet}
	Nutzer klickt auf „Abgeschlossene Semester ein-/ausblenden“.
	& Die Zeilen bereits abgeschlossener Semester werden (wieder) aus"=/eingeblendet.
\end{usecase}

\begin{usecase}{\lA{220}: Studienplan auf Korrektheit überprüfen}
	{Bearbeitungsansicht, Studienplan geöffnet mit Überprüfungsstatus „nicht überprüft“}
	Nutzer klickt auf „Überprüfen“.
	& Ggfs. erscheint ein Ladekreis während längerer Wartezeit. \newline 
	Nach Abschluss der Überprüfung erhält der Plan den Status „korrekt“ oder „fehlerhaft“, was dem Nutzer auch durch eine Notification am Bildschirmrand gemeldet wird. Module, die Konflikte hervorrufen, werden mit einer roten „Fehler“"=Markierung gekennzeichnet. \\
	\hline
	Nutzer kann nun mit der Maus über fehlerhafte Module hovern.
	& Daraufhin erscheint ein Tooltip, das den jeweiligen Konflikt erklärt. \\
	\hline
	Der Nutzer kann Änderungen am Plan vornehmen (\refA{140} bis \refA{170}). 
	& Daraufhin verschwinden die „Fehler“"=Markierungen und der Plan erhält den Status „nicht überprüft“.
\end{usecase}

\begin{usecase}{\lA{230}: Studienplan vervollständigen lassen}
	{Bearbeitungsansicht, Studienplan geöffnet, es können bereits Module im Plan verteilt sein (s. \refA{140} und \refA{150}); Module können Präferenzen haben (\refA{180} und \refA{190})}
	Nutzer klickt auf den Knopf „Plan vervollständigen“.
	& Dem Nutzer erscheint das Vervollständigungs"=Formular, in welchem folgende Daten abgefragt werden:
	\begin{tblitemize}
		\item Zieleigenschaft des vervollständigten Plans:
		\begin{tblitemize}
			\item ECTS"=Minimum
			\item Gewünschte Vertiefungsrichtung
			\item Möglichst schneller Studienabschluss
			\item Möglichst gleichmäßig über alle Semester verteilte ECTS
		\end{tblitemize}
		\item Semesterbegrenzungen für ECTS
		\item Minimale/Maximale Semesterzahl
		\item Angabe der gewünschten Vertiefungsrichtungen
		\item Präferenzen für Module (Like oder Dislike)
	\end{tblitemize} \\
	\hline
	Nutzer gibt geforderte Daten ein und bestätigt.
	& Das System generiert – sofern möglich – einen vollständigen, den Kriterien des Nutzers und des zugrundeliegenden Datensatzes entsprechenden Studienplan. Dabei werden auch Modulpräferenzen, bereits belegte/bestandene sowie eingeplante Module berücksichtigt. \newline
	Im Erfolgsfall erscheint dieser Studienplan dem Nutzer zur Ansicht mitsamt einer entsprechenden Notification am Bildschirmrand. Der vervollständigte Plan hat den Status „korrekt“. Dem Nutzer wird in der Seitenleiste angeboten, den vorgeschlagenen Plan...
	\begin{tblitemize}
		\item zu verwerfen,
		\item zu übernehmen,
		\item oder unter neuem Namen zu speichern.
	\end{tblitemize} \newline
	Im Fehlerfall erscheint eine entspr. Notification am Bildschirmrand. Der Ausgangsplan erscheint und hat den Status „fehlerhaft“; der Nutzer wird wie in \refA{220} über „Fehler“"=Markierungen auf Konflikte hingewiesen. \\
	\hline
	Im Erfolgsfall kann der Nutzer nun den vervollständigten Plan betrachten und durch Klick auf Module im Plan entspr. Info"=Leiste (\refA{130}) anzeigen.
	Er entscheidet sich letztlich zwischen den drei genannten Optionen. 
	& „Verwerfen“: Der Nutzer wird gebeten, das Verwerfen des Vorschlags zu bestätigen. \newline
	„Übernehmen“: Der Vorschlag wird in den Ausgangsplan übernommen. Dem Nutzer erscheint die Bearbeitungsansicht mit dem vervollständigten „korrekten“ Plan. \newline
	„Unter neuem Namen speichern“:  Der Nutzer wird via Popup nach einem Namen für den Vorschlag gefragt. \\
	\hline
	„Verwerfen“: Der Nutzer wählt Bestätigung oder Abbruch. \newline
	„Unter neuem Namen speichern“: Der Nutzer gibt den Namen ein. 
	& „Verwerfen“: Im Falle der Bestätigung kehrt der Nutzer in die ursprüngliche Bearbeitungsansicht mit dem Ausgangsplan im alten Status zurück. Lehnt der Nutzer dies ab, so kehrt er zur Vorschlagsansicht zurück, wo ihm wieder die drei Optionen angeboten werden. \newline
	„Unter neuem Namen speichern“: Der Vorschlag wird in einen neuen Plan mit angegebenem Namen gespeichert. Dem Nutzer erscheint die Bearbeitungsansicht mit dem neuen, vervollständigten Plan. Dieser Plan hat den Status „korrekt“.
\end{usecase}

\begin{usecase}{\lA{240}: Schließen des Studienplans mit Wechsel zur Hauptansicht}
	{Bearbeitungsansicht, Studienplan geöffnet}
	Nutzer betätigt den Schalter zur Hauptansicht.
	& Die Hauptansicht erscheint, der zuvor bearbeitete Studienplan trägt nun ggfs. auch einen neuen Namen (\refA{60}) und einen neuen Überprüfungsstatus (\refA{220}).
\end{usecase}


\bigskip

% reset changes
\renewcommand{\arraystretch}{1.0}
\setlength{\LTpre}{\bigskipamount}
\setlength{\LTpost}{\bigskipamount}