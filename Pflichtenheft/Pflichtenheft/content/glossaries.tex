%
% % Automatisch generiertes Glossar
%
%\glsaddall % das sorgt dafür, dass alles Glossareinträge gedruckt werden, nicht nur die verwendeten. Das sollte nicht nötig sein!
%
% % Glossareinträge
%
\newglossaryentry{Rest}
{
	name=REST,
	description={Abk. für Representational State Transfer, Programmierparadigma für \glspl{Webservice} auf Basis des HTTP-Protokolls}
}

\newglossaryentry{Modul}
{
	name=Modul,
	description={ist bei Bachelor- und Master-Studiengängen eine Lehreinheit, die aus einer oder mehreren Lehrveranstaltungen mit einem gemeinsamen Lernziel besteht},
	plural=Module
}

\newglossaryentry{ECTS-Punkte}
{
	name=ECTS-Punkte,
	description={Leistungspunkte, die für ein erfolgreich absolviertes \gls{Modul} von der Hochschule auf Basis des ECTS-Punktesystems vergibt werden, und mit denen der Arbeitsaufwand „gemessen“ wird},
}

\newglossaryentry{Generierungs-Tool}
{
	name=Generierungs-Tool,
	description={Tool, für die automatische Erstellung bzw. Vervollständigung von Studienpläne},
}


\newglossaryentry{Webservice}
{
	name=Webservice,
	plural=Webservices,
	description={Softwareanwendung, die über ein Netzwerk bereitgestellt wird}
}

\newglossaryentry{Plug-In-Paket}
{
	name=Plug-In-Paket,
	plural=Plug-In-Pakete,
	description={Paket bestehend aus mehreren \glspl{Plug-In}}
}

\newglossaryentry{iMage}
{
	name={iMage},
	description={Bildbearbeitungssoftware der Firma SWT. Bietet im Basispaket nur Funktionalitäten zur Skalierung und Drehung von Bildern}
}

\newglossaryentry{Internetbrowser}
{
	name={Internetbrowser},
	description={Programm, mit dem Websites gefunden, gelesen und verwaltet werden können, mit aktiviertem JavaScript}
}

\newglossaryentry{Online-Shop}
{
	name={Online-Shop},
	description={Internetseite, die Produkte zum Kauf anbietet}
}